%
% File acl2017.tex
%
%% Based on the style files for ACL-2015, with some improvements
%%  taken from the NAACL-2016 style
%% Based on the style files for ACL-2014, which were, in turn,
%% based on ACL-2013, ACL-2012, ACL-2011, ACL-2010, ACL-IJCNLP-2009,
%% EACL-2009, IJCNLP-2008...
%% Based on the style files for EACL 2006 by 
%%e.agirre@ehu.es or Sergi.Balari@uab.es
%% and that of ACL 08 by Joakim Nivre and Noah Smith

\documentclass[11pt,a4paper]{article}
\usepackage[hyperref]{acl2017}
\usepackage{times}
\usepackage{latexsym}

\usepackage{url}

\aclfinalcopy % Uncomment this line for the all SemEval submissions
%\def\aclpaperid{***} %  Enter the acl Paper ID here

%\setlength\titlebox{5cm}
% You can expand the titlebox if you need extra space
% to show all the authors. Please do not make the titlebox
% smaller than 5cm (the original size); we will check this
% in the camera-ready version and ask you to change it back.

\newcommand\BibTeX{B{\sc ib}\TeX}

%Title format for system description papers by task participants
\title{Duluth at SemEval-2017 Task 6:  Language Models in Humor Recognition}
%Title format for task description papers by task organizers
%\title{SemEval-2017 Task [TaskNumber]:  [Task Name]}


\author{Xinru Yan \\
  Department of Computer Science \\ University of Minnesota Duluth \\ Duluth, MN, 55812 USA \\
  {\tt yanxx418@d.umn.edu} \\\And
  Ted Pedersen \\
  Department of Computer Science \\ University of Minnesota Duluth \\ Duluth, MN, 55812 USA \\
  {\tt tpederse@d.umn.edu} \\}

\date{}

\begin{document}
\maketitle
\begin{abstract}
  This paper describes the Duluth system that participated in SemEval-2017 Task 6 \#HashtagWars: Learning a Sense of Humor. The system completed Task A and Task B using ngram language models. This paper includes the results of our system with several post-evaluation runs. Our system is ranked highly in both tasks during evaluation.
\end{abstract}

\section{Introduction}
Humor is considered to be a human-only trait. It is one of the most amusing and mystifying human activities. Since humor represents human uniqueness and intelligence to some extent, it has continuously drawn attention in different research areas such as linguistics, psychology, philosophy and Computer Science. In Computer Science, relevant theories derived from those fields have formed a relatively young area of study - computational humor ~\cite{Recognizing:Humor:On:Twitter}. Humor has not been addressed broadly in current computational research. Many studies have developed decent systems to produce humor ~\cite{ozbal2012computational}. However, humor recognition is essentially a more challenging and fun task. In order to address these concerns, we develop methods that consider its continuous and subjective nature of humor by learning a sense of humor from a particular source including humorous tweets submitted to a famous Comedy Central TV show, \textit{@midnight with Chris Hardwick}. This research is associated with the SemEval-2017 Task 6. The essential purpose of this research is to make computers understand humor in a subjective way and learn the continuity aspect of humor from a peculiar dataset.\\
In addition to humor related work, we have also benefited from some research on memorability in developing humor recognition methods.

\section{Background}
Language models are a straightforward way to collect set of rules by utilizing the fact that words do not appear in an arbitrary order, which means we can learn a lot from a word and its neighbors ~\cite{de2011natural}. A statistical language model is a model that computes the probability of a sequence of words or an upcoming word ~\cite{JM}.\\
Below are two examples of language modeling:
To compute the probability of a sequence of words $W$ given the sequence $(w_{1},w_{2},...w_{n})$, we have:
\begin{equation}
P(W) = P(w_{1},w_{2},...w_{n})
\end{equation}
To compute the probability of an upcoming word $w3$ given the sequence $(w_{1},w_{2})$, the language model gives us the following probability:
\begin{equation}
P(w_{3}|w_{1},w_{2})
\end{equation}
The idea of word prediction with probabilistic models is called N-gram models, which predict the upcoming word from the previous N-1 words. An N-gram is a contiguous sequence of N words: a unigram is a single word, a bigram is a two-word sequence of words and a trigram is a three-word sequence of words. For example, in tweet "tears in Ramen \#SingleLifeIn3Words", "tears", "in", "Ramen" and "\#SingleLifeIn3Words" are unigrams; "tears in", "in Ramen" and "Ramen \#SingleLifeIn3Words" are bigrams and "tears in Ramen" and "in Ramen \#SingleLifeIn3Words" are trigrams.\\
In the study on how phrasing affects memorability, in order to analyze the characteristics of memorable quotes, researchers take
language model approach to investigate distinctiveness feature and employ syntactic measures on the data to evaluate generality feature ~\cite{hello}. Specifically, in favor of evaluating how distinctive a quote is, they evaluate its likelihood with the respect of the “common language” model which consists of the newswire sections of the Brown corpus. They employ six add-1 smoothed language models–unigram, bigram, trigram word language models and unigram, bigram, trigram Part of Speech (POS) language models–on the “common language”model. The idea of using language models to assess the memorability of a quote is suitable for our purpose of detecting how humorous a twitter is. Except for using funny tweets provided by the task to train ngram language models, our system also trained ngram language models on English news data in order to evaluate how distictive, in this case, how funny, a tweet is comparing to news.\\ For our purpose, we trained unigrams, bigrams and trigrams on both sets of training data.

\section{Method}
Our system estimates tweet probability using ngram models. Specifically, it solves the given problem in four steps:
\begin{enumerate}
\item Corpus preparing and pre-processing: Collect all training data files to form one training corpus. Pre-processing includes filtering and tokenization.
\item Language model training: Build n-gram language models by feeding the corpus to KenLM Language Model Toolkit ~\cite{Heafield-estimate}. 
\item Tweet scoring: Get log probability for each tweet based on the trained ngram language model.
\item Tweet prediction: According to the log probability
\begin{itemize}
\item Given two tweets, comparing two tweets and predicting which one is funnier (for subtask A)
\item Given a set of tweets associated with one specific hashtag, ranking tweets from the funnest to the least funny (for subtask B)
\end{itemize}
\end{enumerate}

\subsection{Corpus preparing and pre-processing}
In our system, we used two distinct sets of training data: the tweets data and the news data. The tweets data is provided by the SemEval task. It consists of 106 hashtag files, about 21580 tokens. In addition, we collected in total of 6.2 GB of English news data, about 2002655 tokens, from the News Commentary Corpus and the News Craw Corpus from yeas of 2008, 2010 and 2011 \footnote{http://www.statmt.org/wmt11/translation-task.html\#download}.   
\subsubsection{Preparing}
To prepare the tweets data, the system takes in total of 106 hashtag files, which includes both trial\_dir and train\_dir from the task, and put all tweets in one plain text file to form the tweet training corpus. Each tweet is on its own line. Be aware that during the development phase of the system, we trained the language model on the train\_dir data and tested it on the trial\_dir data. \\
For the news data, the system reads in all the sentences from the news files and again, put them in one plain text file to form the news training corpus. Each sentence takes its own line.
\subsubsection{Pre-processing}
In general, the pre-processing consists of two steps: filtering and tokenization. The filtering step is mainly for the tweet training corpus. Also, we applied various filtering and tokenziation combinations on experiments to determine the best settings (see section 4). 
\begin{itemize}
\item Filtering: the filtering process includes removing following elements from the :
\begin{itemize}
\item URLs
\item Twitter user names with symbol @ indicating the user name
\item Hashtags with symbol \# indicating the topic of the tweet
\end{itemize}
\item Tokenization: For both training data sets we splitted text by space and punctuation marks
\end{itemize}
\subsection{Language Model Training}
Once we have the corpus ready, we use the KenLM Toolkit to train the n-gram language models on the corpus. Language models are estimated from the corpus using modified Kneser-Ney smoothing without pruning. KenLM reads in a plain text file and generates language models in arpa format. We trained three different language models -- unigrams, bigrams and trigrams -- for both training data sets. KenLM also implements back off technique, which simply means it applies the lower order ngram's probability along with its back-off weights if the ngram is not found. Instead of using the real probability of the ngram, KenLM applies base 10 logarithm scheme. Here is an example of the trigram model we trained on the tweets data:

\begin{tabular}{c}
\hline
ngram 1=21580 \\
ngram 2=60624 \\
ngram 3=73837 \\
unigram:\\
-4.8225346  \textless unk\textgreater  0 \\
0  \textless s\textgreater  -0.47505832 \\
-1.4503417  \textless /s\textgreater  0 \\
-4.415446  Donner  -0.12937292 \\
-3.4745252  Party  -0.09994553 \\
...\\
\hline
bigrams:\\
...\\
-0.9799023  Drilling Gulf -0.024524588\\
-3.9588327  of Mexicobe -0.024524588\\
...\\
\hline
trigrams:\\
...\\
-1.171928 I'll start thinking\\
-1.2377753  thinking he cares\\
...\\
\hline
\end{tabular}

Each ngram line starts with the base 10 logarithm probability of that ngram, followed by the ngram which consists of n words. The base 10 logarithm of the back-off weight for the ngram is followed after optionally. In this trigram language model trained on the tweets data, there are 73837 trigrams in total from the tweet training corpus. Notice that there are three "special" words in a language model: the beginning of a sentence denoted by \textless s\textgreater, the end of a sentence denoted by \textless /s\textgreater and the out of vocabulary word denoted by \textless unk\textgreater. In order to be able to handle the unknown words to estimate the probability of a tweet more accurately, in all our experiments we keep the \textless unk\textgreater word in the language model. To figure out the best setting of language model for both tasks, we experiment using the language model with and without sentence boundaries.

\subsection{Tweet Scoring}
After training the ngram model, the next step is scoring. For each hashtag file that needs to be evaluated, based on the trained ngram language model, the system assigns a base 10 log probability for each tweet in the hashtag file. Here is an example of scored tweet from hashtag file Bad\_Job\_In\_5\_Words.tsv based on the ngram language model trained on the tweets data:\\
705511149970726912  The host of Singled Out \#BadJobIn5Words @midnight -19.923433303833008
705538894415003648  Donut receipt maker and sorter  \#BadJobIn5Words @midnight -27.67446517944336
\subsection{Tweet Prediction}
The system sorts tweets for each hashtag file based on their score in descending order, meaning the most probable one is listed on the top. For Task A, given a hashtag file, the system goes through the sorted list of tweets, compare each pair of tweets and produces a tsv format file as the task asks for. For each tweet pair twee\_1 and tweet\_2, if tweet\_1 has higher score, system outputs tweet\_ids for the pair followed by "1" and followed by "0" otherwise. For Task B, given a hashtag file, the systm simply outputs tweet\_ids in the order of the sorted list. 
\section{Experiments and Results}
Includes result table
\section{Discussion and Future Work}
Includes using ML technique in the future
\section{References} 

\subsection{The Ruler}
The ACL 2017 style defines a printed ruler which should be presented in the
version submitted for review.  The ruler is provided in order that
reviewers may comment on particular lines in the paper without
circumlocution.  If you are preparing a document without the provided
style files, please arrange for an equivalent ruler to
appear on the final output pages.  The presence or absence of the ruler
should not change the appearance of any other content on the page.  The
camera ready copy should not contain a ruler. (\LaTeX\ users may uncomment
the {\small\verb|\aclfinalcopy|} command in the document preamble.)  

Reviewers: note that the ruler measurements do not align well with
lines in the paper -- this turns out to be very difficult to do well
when the paper contains many figures and equations, and, when done,
looks ugly. In most cases one would expect that the approximate
location will be adequate, although you can also use fractional
references ({\em e.g.}, the first paragraph on this page ends at mark $114.5$).

\subsection{Electronically-available resources}

ACL provides this description in \LaTeX2e{} ({\small\tt acl2017.tex}) and PDF
format ({\small\tt acl2017.pdf}), along with the \LaTeX2e{} style file used to
format it ({\small\tt acl2017.sty}) and an ACL bibliography style ({\small\tt acl\_natbib.bst})
and example bibliography ({\small\tt acl2017.bib}).
These files are all available at
{\small\tt acl2017.org/index.php?article\_id=9}. We
strongly recommend the use of these style files, which have been
appropriately tailored for the ACL 2017 proceedings.

\subsection{Format of Electronic Manuscript}
\label{sect:pdf}

For the production of the electronic manuscript you must use Adobe's
Portable Document Format (PDF). PDF files are usually produced from
\LaTeX\ using the \textit{pdflatex} command. If your version of
\LaTeX\ produces Postscript files, you can convert these into PDF
using \textit{ps2pdf} or \textit{dvipdf}. On Windows, you can also use
Adobe Distiller to generate PDF.

Please make sure that your PDF file includes all the necessary fonts
(especially tree diagrams, symbols, and fonts with Asian
characters). When you print or create the PDF file, there is usually
an option in your printer setup to include none, all or just
non-standard fonts.  Please make sure that you select the option of
including ALL the fonts. \textbf{Before sending it, test your PDF by
  printing it from a computer different from the one where it was
  created.} Moreover, some word processors may generate very large PDF
files, where each page is rendered as an image. Such images may
reproduce poorly. In this case, try alternative ways to obtain the
PDF. One way on some systems is to install a driver for a postscript
printer, send your document to the printer specifying ``Output to a
file'', then convert the file to PDF.

It is of utmost importance to specify the \textbf{A4 format} (21 cm
x 29.7 cm) when formatting the paper. When working with
{\tt dvips}, for instance, one should specify {\tt -t a4}.
Or using the command \verb|\special{papersize=210mm,297mm}| in the latex
preamble (directly below the \verb|\usepackage| commands). Then using 
{\tt dvipdf} and/or {\tt pdflatex} which would make it easier for some.

Print-outs of the PDF file on A4 paper should be identical to the
hardcopy version. If you cannot meet the above requirements about the
production of your electronic submission, please contact the
publication chairs as soon as possible.

\subsection{Layout}
\label{ssec:layout}

Format manuscripts two columns to a page, in the manner these
instructions are formatted. The exact dimensions for a page on A4
paper are:

\begin{itemize}
\item Left and right margins: 2.5 cm
\item Top margin: 2.5 cm
\item Bottom margin: 2.5 cm
\item Column width: 7.7 cm
\item Column height: 24.7 cm
\item Gap between columns: 0.6 cm
\end{itemize}

\noindent Papers should not be submitted on any other paper size.
 If you cannot meet the above requirements about the production of 
 your electronic submission, please contact the publication chairs 
 above as soon as possible.

\subsection{Fonts}

For reasons of uniformity, Adobe's {\bf Times Roman} font should be
used. In \LaTeX2e{} this is accomplished by putting

\begin{quote}
\begin{verbatim}
\usepackage{times}
\usepackage{latexsym}
\end{verbatim}
\end{quote}
in the preamble. If Times Roman is unavailable, use {\bf Computer
  Modern Roman} (\LaTeX2e{}'s default).  Note that the latter is about
  10\% less dense than Adobe's Times Roman font.

\begin{table}[h]
\begin{center}
\begin{tabular}{|l|rl|}
\hline \bf Type of Text & \bf Font Size & \bf Style \\ \hline
paper title & 15 pt & bold \\
author names & 12 pt & bold \\
author affiliation & 12 pt & \\
the word ``Abstract'' & 12 pt & bold \\
section titles & 12 pt & bold \\
document text & 11 pt  &\\
captions & 11 pt & \\
abstract text & 10 pt & \\
bibliography & 10 pt & \\
footnotes & 9 pt & \\
\hline
\end{tabular}
\end{center}
\caption{\label{font-table} Font guide. }
\end{table}

\subsection{The First Page}
\label{ssec:first}

Center the title, author's name(s) and affiliation(s) across both
columns. Do not use footnotes for affiliations. Do not include the
paper ID number assigned during the submission process. Use the
two-column format only when you begin the abstract.

{\bf Title}: Place the title centered at the top of the first page, in
a 15-point bold font. (For a complete guide to font sizes and styles,
see Table~\ref{font-table}) Long titles should be typed on two lines
without a blank line intervening. Approximately, put the title at 2.5
cm from the top of the page, followed by a blank line, then the
author's names(s), and the affiliation on the following line. Do not
use only initials for given names (middle initials are allowed). Do
not format surnames in all capitals ({\em e.g.}, use ``Mitchell'' not
``MITCHELL'').  Do not format title and section headings in all
capitals as well except for proper names (such as ``BLEU'') that are
conventionally in all capitals.  The affiliation should contain the
author's complete address, and if possible, an electronic mail
address. Start the body of the first page 7.5 cm from the top of the
page.

The title, author names and addresses should be completely identical
to those entered to the electronical paper submission website in order
to maintain the consistency of author information among all
publications of the conference. If they are different, the publication
chairs may resolve the difference without consulting with you; so it
is in your own interest to double-check that the information is
consistent.

{\bf Abstract}: Type the abstract at the beginning of the first
column. The width of the abstract text should be smaller than the
width of the columns for the text in the body of the paper by about
0.6 cm on each side. Center the word {\bf Abstract} in a 12 point bold
font above the body of the abstract. The abstract should be a concise
summary of the general thesis and conclusions of the paper. It should
be no longer than 200 words. The abstract text should be in 10 point font.

{\bf Text}: Begin typing the main body of the text immediately after
the abstract, observing the two-column format as shown in 
the present document. Do not include page numbers.

{\bf Indent}: When starting a new paragraph. Use 11 points for text and 
subsection headings, 12 points for section headings and 15 points for
the title. 

\begin{table}
\centering
\small
\begin{tabular}{cc}
\begin{tabular}{|l|l|}
\hline
{\bf Command} & {\bf Output}\\\hline
\verb|{\"a}| & {\"a} \\
\verb|{\^e}| & {\^e} \\
\verb|{\`i}| & {\`i} \\ 
\verb|{\.I}| & {\.I} \\ 
\verb|{\o}| & {\o} \\
\verb|{\'u}| & {\'u}  \\ 
\verb|{\aa}| & {\aa}  \\\hline
\end{tabular} & 
\begin{tabular}{|l|l|}
\hline
{\bf Command} & {\bf  Output}\\\hline
\verb|{\c c}| & {\c c} \\ 
\verb|{\u g}| & {\u g} \\ 
\verb|{\l}| & {\l} \\ 
\verb|{\~n}| & {\~n} \\ 
\verb|{\H o}| & {\H o} \\ 
\verb|{\v r}| & {\v r} \\ 
\verb|{\ss}| & {\ss} \\\hline
\end{tabular}
\end{tabular}
\caption{Example commands for accented characters, to be used in, {\em e.g.}, \BibTeX\ names.}\label{tab:accents}
\end{table}

\subsection{Sections}

{\bf Headings}: Type and label section and subsection headings in the
style shown on the present document.  Use numbered sections (Arabic
numerals) in order to facilitate cross references. Number subsections
with the section number and the subsection number separated by a dot,
in Arabic numerals. Do not number subsubsections.

\begin{table*}
\centering
\begin{tabular}{lll}
  output & natbib & previous ACL style files\\
  \hline
  \citep{Gusfield:97} & \verb|\citep| & \verb|\cite| \\
  \citet{Gusfield:97} & \verb|\citet| & \verb|\newcite| \\
  \citeyearpar{Gusfield:97} & \verb|\citeyearpar| & \verb|\shortcite| \\
\end{tabular}
\caption{Citation commands supported by the style file.
  The citation style is based on the natbib package and
  supports all natbib citation commands.
  It also supports commands defined in previous ACL style files
  for compatibility.
  }
\end{table*}

{\bf Citations}: Citations within the text appear in parentheses
as~\cite{Gusfield:97} or, if the author's name appears in the text
itself, as Gusfield~\shortcite{Gusfield:97}.
Using the provided \LaTeX\ style, the former is accomplished using
{\small\verb|\cite|} and the latter with {\small\verb|\shortcite|} or {\small\verb|\newcite|}.  Collapse multiple citations as in~\cite{Gusfield:97,Aho:72}; this is accomplished with the provided style using commas within the {\small\verb|\cite|} command, {\em e.g.}, {\small\verb|\cite{Gusfield:97,Aho:72}|}.  
Append lowercase letters to the year in cases of ambiguities.  
 Treat double authors as
in~\cite{Aho:72}, but write as in~\cite{Chandra:81} when more than two
authors are involved. Collapse multiple citations as
in~\cite{Gusfield:97,Aho:72}. Also refrain from using full citations
as sentence constituents.

\penalty -5000

We suggest that instead of
\begin{quote}
  ``\cite{Gusfield:97} showed that ...''
\end{quote}
you use
\begin{quote}
``Gusfield \shortcite{Gusfield:97}   showed that ...''
\end{quote}

If you are using the provided \LaTeX{} and Bib\TeX{} style files, you
can use the command \verb|\citet| (cite in text)
to get ``author (year)'' citations.

If the Bib\TeX{} file contains DOI fields, the paper
title in the references section will appear as a hyperlink
to the DOI, using the hyperref \LaTeX{} package.
To disable the hyperref package, load the style file
with the \verb|nohyperref| option:
\verb|\usepackage[nohyperref]{acl2017}|

\textbf{Digital Object Identifiers}:  As part of our work to make ACL
materials more widely used and cited outside of our discipline, ACL
has registered as a CrossRef member, as a registrant of Digital Object
Identifiers (DOIs), the standard for registering permanent URNs for
referencing scholarly materials.  As of 2017, we are requiring all
camera-ready references to contain the appropriate DOIs (or as a
second resort, the hyperlinked ACL Anthology Identifier) to all cited
works.  Thus, please ensure that you use Bib\TeX records that contain
DOI or URLs for any of the ACL materials that you reference.
Appropriate records should be found for most materials in the current
ACL Anthology at \url{http://aclanthology.info/}.

As examples, we cite \cite{P16-1001} to show you how papers with a DOI
will appear in the bibliography.  We cite \cite{C14-1001} to show how
papers without a DOI but with an ACL Anthology Identifier will appear
in the bibliography.  

As reviewing will be double-blind, the submitted version of the papers
should not include the authors' names and affiliations. Furthermore,
self-references that reveal the author's identity, {\em e.g.},
\begin{quote}
``We previously showed \cite{Gusfield:97} ...''  
\end{quote}
should be avoided. Instead, use citations such as 
\begin{quote}
``\citeauthor{Gusfield:97} \shortcite{Gusfield:97}
previously showed ... ''
\end{quote}

\textbf{Please do not use anonymous citations} and do not include
acknowledgements when submitting your papers. Papers that do not
conform to these requirements may be rejected without review.

\textbf{References}: Gather the full set of references together under
the heading {\bf References}; place the section before any Appendices,
unless they contain references. Arrange the references alphabetically
by first author, rather than by order of occurrence in the text.
Provide as complete a citation as possible, using a consistent format,
such as the one for {\em Computational Linguistics\/} or the one in the 
{\em Publication Manual of the American 
Psychological Association\/}~\cite{APA:83}.  Use of full names for
authors rather than initials is preferred.  A list of abbreviations
for common computer science journals can be found in the ACM 
{\em Computing Reviews\/}~\cite{ACM:83}.

The \LaTeX{} and Bib\TeX{} style files provided roughly fit the
American Psychological Association format, allowing regular citations, 
short citations and multiple citations as described above.

{\bf Appendices}: Appendices, if any, directly follow the text and the
references (but see above).  Letter them in sequence and provide an
informative title: {\bf Appendix A. Title of Appendix}.

\subsection{Footnotes}

{\bf Footnotes}: Put footnotes at the bottom of the page and use 9
points text. They may be numbered or referred to by asterisks or other
symbols.\footnote{This is how a footnote should appear.} Footnotes
should be separated from the text by a line.\footnote{Note the line
separating the footnotes from the text.}

\subsection{Graphics}

{\bf Illustrations}: Place figures, tables, and photographs in the
paper near where they are first discussed, rather than at the end, if
possible.  Wide illustrations may run across both columns.  Color
illustrations are discouraged, unless you have verified that  
they will be understandable when printed in black ink.

{\bf Captions}: Provide a caption for every illustration; number each one
sequentially in the form:  ``Figure 1. Caption of the Figure.'' ``Table 1.
Caption of the Table.''  Type the captions of the figures and 
tables below the body, using 11 point text.

\subsection{Accessibility}
\label{ssec:accessibility}

In an effort to accommodate the color-blind (as well as those printing
to paper), grayscale readability for all accepted papers will be
encouraged.  Color is not forbidden, but authors should ensure that
tables and figures do not rely solely on color to convey critical
distinctions.
Here we give a simple criterion on your colored figures, if your paper has to be printed in black and white, then you must assure that every curves or points in your figures can be still clearly distinguished.

% Min: no longer used as of ACL 2017, following ACL exec's decision to
% remove this extra workflow that was not executed much.
% BEGIN: remove
%% \section{XML conversion and supported \LaTeX\ packages}

%% Following ACL 2014 we will also we will attempt to automatically convert 
%% your \LaTeX\ source files to publish papers in machine-readable 
%% XML with semantic markup in the ACL Anthology, in addition to the 
%% traditional PDF format.  This will allow us to create, over the next 
%% few years, a growing corpus of scientific text for our own future research, 
%% and picks up on recent initiatives on converting ACL papers from earlier 
%% years to XML. 

%% We encourage you to submit a ZIP file of your \LaTeX\ sources along
%% with the camera-ready version of your paper. We will then convert them
%% to XML automatically, using the LaTeXML tool
%% (\url{http://dlmf.nist.gov/LaTeXML}). LaTeXML has \emph{bindings} for
%% a number of \LaTeX\ packages, including the ACL 2017 stylefile. These
%% bindings allow LaTeXML to render the commands from these packages
%% correctly in XML. For best results, we encourage you to use the
%% packages that are officially supported by LaTeXML, listed at
%% \url{http://dlmf.nist.gov/LaTeXML/manual/included.bindings}
% END: remove

\section{Translation of non-English Terms}

It is also advised to supplement non-English characters and terms
with appropriate transliterations and/or translations
since not all readers understand all such characters and terms.
Inline transliteration or translation can be represented in
the order of: original-form transliteration ``translation''.

\section{Length of Submission}
\label{sec:length}

The ACL 2017 main conference accepts submissions of long papers and
short papers.
 Long papers may consist of up to eight (8) pages of
content plus unlimited pages for references. Upon acceptance, final
versions of long papers will be given one additional page -- up to nine (9)
pages of content plus unlimited pages for references -- so that reviewers' comments
can be taken into account. Short papers may consist of up to four (4)
pages of content, plus unlimited pages for references. Upon
acceptance, short papers will be given five (5) pages in the
proceedings and unlimited pages for references. 

For both long and short papers, all illustrations and tables that are part
of the main text must be accommodated within these page limits, observing
the formatting instructions given in the present document. Supplementary
material in the form of appendices does not count towards the page limit.

However, note that supplementary material should be supplementary
(rather than central) to the paper, and that reviewers may ignore
supplementary material when reviewing the paper (see Appendix
\ref{sec:supplemental}). Papers that do not conform to the specified
length and formatting requirements are subject to be rejected without
review.

Workshop chairs may have different rules for allowed length and
whether supplemental material is welcome. As always, the respective
call for papers is the authoritative source.

\section*{Acknowledgments}

The acknowledgments should go immediately before the references.  Do
not number the acknowledgments section. Do not include this section
when submitting your paper for review.

% include your own bib file like this:
%\bibliographystyle{acl}
%\bibliography{acl2017}
\bibliography{semeval2017}
\bibliographystyle{acl_natbib}

\appendix

\section{Supplemental Material}
\label{sec:supplemental}
ACL 2017 also encourages the submission of supplementary material
to report preprocessing decisions, model parameters, and other details
necessary for the replication of the experiments reported in the 
paper. Seemingly small preprocessing decisions can sometimes make
a large difference in performance, so it is crucial to record such
decisions to precisely characterize state-of-the-art methods.

Nonetheless, supplementary material should be supplementary (rather
than central) to the paper. {\bf Submissions that misuse the supplementary 
material may be rejected without review.}
Essentially, supplementary material may include explanations or details
of proofs or derivations that do not fit into the paper, lists of
features or feature templates, sample inputs and outputs for a system,
pseudo-code or source code, and data. (Source code and data should
be separate uploads, rather than part of the paper).

The paper should not rely on the supplementary material: while the paper
may refer to and cite the supplementary material and the supplementary material will be available to the
reviewers, they will not be asked to review the
supplementary material.

Appendices ({\em i.e.} supplementary material in the form of proofs, tables,
or pseudo-code) should come after the references, as shown here. Use
\verb|\appendix| before any appendix section to switch the section
numbering over to letters.

\section{Multiple Appendices}
\dots can be gotten by using more than one section. We hope you won't
need that.

\end{document}
