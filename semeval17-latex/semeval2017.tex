%
% File acl2017.tex
%
%% Based on the style files for ACL-2015, with some improvements
%%  taken from the NAACL-2016 style
%% Based on the style files for ACL-2014, which were, in turn,
%% based on ACL-2013, ACL-2012, ACL-2011, ACL-2010, ACL-IJCNLP-2009,
%% EACL-2009, IJCNLP-2008...
%% Based on the style files for EACL 2006 by 
%%e.agirre@ehu.es or Sergi.Balari@uab.es
%% and that of ACL 08 by Joakim Nivre and Noah Smith

\documentclass[11pt,a4paper]{article}
\usepackage[hyperref]{acl2017}
\usepackage{times}
\usepackage{latexsym}

\usepackage{url}

\aclfinalcopy % Uncomment this line for the all SemEval submissions
%\def\aclpaperid{***} %  Enter the acl Paper ID here

%\setlength\titlebox{5cm}
% You can expand the titlebox if you need extra space
% to show all the authors. Please do not make the titlebox
% smaller than 5cm (the original size); we will check this
% in the camera-ready version and ask you to change it back.

\newcommand\BibTeX{B{\sc ib}\TeX}

%Title format for system description papers by task participants
\title{Duluth at SemEval-2017 Task 6:  Language Models in Humor Detection}
%Title format for task description papers by task organizers
%\title{SemEval-2017 Task [TaskNumber]:  [Task Name]}


\author{Xinru Yan \\
  Department of Computer Science \\ University of Minnesota Duluth \\ Duluth, MN, 55812 USA \\
  {\tt yanxx418@d.umn.edu} \\\And
  Ted Pedersen \\
  Department of Computer Science \\ University of Minnesota Duluth \\ Duluth, MN, 55812 USA \\
  {\tt tpederse@d.umn.edu} \\}

\date{}

\begin{document}
\maketitle
\begin{abstract}
  This paper describes the Duluth system that participated in SemEval-2017 Task 6 \#HashtagWars: Learning a Sense of Humor. The system completed Task A and Task B using N-gram language models, ranking well during evaluation. This paper includes the evaluation results of our system with several post-evaluation runs. 
\end{abstract}

\section{Introduction}
Since humor represents human uniqueness and intelligence to some extent, it has continuously drawn attention in different research areas such as linguistics, psychology, philosophy and Computer Science. In Computer Science, relevant theories derived from those fields have formed a relatively young area of study - computational humor \cite{Recognizing:Humor:On:Twitter}. Humor has not been addressed broadly in current computational research. Many studies have developed decent systems to produce humor \cite{ozbal2012computational}. However, humor detection is essentially a more challenging and fun problem. SemEval-2017 Task 6 focuses on humor detection by asking participants to develop systems that learn a sense of humor from the Comedy Central TV show, \textit{@midnight with Chris Hardwick}. Our system, Duluth, applies language model approach to detect humor by training N-gram language models on two sets of training data, the tweets data and the news data.\\


\section{Background}
\textbf{Language models}(LMs) are a straightforward way to collect set of rules by utilizing the fact that words do not appear in an arbitrary order, which means we can some useful information from a word and its neighbors ~\cite{de2011natural}. A statistical language model is a model that computes the probability of a sequence of words or an upcoming word ~\cite{JM}. Below are two examples of language modeling:
To compute the probability of a sequence of words $W$ given the sequence $(w_{1},w_{2},...w_{n})$, we have:
\begin{equation}
P(W) = P(w_{1},w_{2},...w_{n})
\end{equation}
To compute the probability of an upcoming word $w3$ given the sequence $(w_{1},w_{2})$, the language model gives us the following probability:
\begin{equation}
P(w_{3}|w_{1},w_{2})
\end{equation}
The idea of word prediction with probabilistic models is called N-gram models, which predict the upcoming word from the previous N-1 words. An N-gram is a contiguous sequence of N words: a unigram is a single word, a bigram is a two-word sequence of words and a trigram is a three-word sequence of words. For example, in tweet "tears in Ramen \#SingleLifeIn3Words", "tears", "in", "Ramen" and "\#SingleLifeIn3Words" are unigrams; "tears in", "in Ramen" and "Ramen \#SingleLifeIn3Words" are bigrams and "tears in Ramen" and "in Ramen \#SingleLifeIn3Words" are trigrams.\\
When we use for example, trigram LM, to predict the conditional probability of the next word, we are thus making the following approximation:
\begin{equation}
P(w_n|w_1^{n-1})\approx P(w_n|w_{n-2}, w_{n-1})
\end{equation}\\
This assumption that the probability of a word depends only on a small number of previous words is called \textbf{Markov} assumption. According to Markov assumption, here we show the general equation for computing the probability of a complete word sequence using trigram LM:
\begin{equation}
P(w_1^n)\approx \prod_{k=1}^{n} P(w_k|w_{k-2}, w_{k-1})
\end{equation}\\
In the study on how phrasing affects memorability, in order to analyze the characteristics of memorable quotes, researchers take
language model approach to investigate distinctiveness feature and employ syntactic measures on the data to evaluate generality feature ~\cite{hello}. Specifically, in favor of evaluating how distinctive a quote is, they evaluate its likelihood with the respect of the “common language” model which consists of the newswire sections of the Brown corpus. They employ six add-1 smoothed language models–unigram, bigram, trigram word language models and unigram, bigram, trigram Part of Speech (POS) language models–on the “common language” model. They come to the conclusion that movie quotes which are less like the "common language" are more memorable. The idea of using language models to assess the memorability of a quote is suitable for our purpose of detecting how humorous a twitter is. Except for using tweets provided by the task to train N-gram LMs, our system also trained N-gram LMs on English news data in order to evaluate how distictive, in this case, how funny, a tweet is comparing to the "common language"--news. Tweets that were more like the tweets model, or less like the news model, were ranked as being more funny. For our purpose, we trained bigram LM and trigram LM on both sets of training data.

\section{Method}
Our system estimates tweet probability using N-gram LMs. Specifically, it solves the given two subtasks in four steps:
\begin{enumerate}
\item Corpus preparing and pre-processing: Collect all training data files to form one training corpus. Pre-processing includes filtering and tokenization.
\item Language model training: Build n-gram language models by feeding the corpus to KenLM Language Model Toolkit ~\cite{Heafield-estimate}. 
\item Tweet scoring: Get log probability for each tweet based on the trained N-gram language model.
\item Tweet prediction: According to the log probability
\begin{itemize}
\item Subtask A -- Given two tweets, comparing them and predicting which one is funnier. 
\item Subtask B -- Given a set of tweets associated with one hashtag, ranking tweets from the funnest to the least funny.
\end{itemize}
\end{enumerate}

\subsection{Corpus preparing and pre-processing}
In our system, we used two distinct sets of training data: the tweets data and the news data. The tweets data is provided by the SemEval task. It consists of 106 hashtag files, about 21,580 tokens. In addition, we collected in total of 6.2 GB of English news data, about 2,002,655 tokens, from the News Commentary Corpus and the News Crawl Corpus from yeas of 2008, 2010 and 2011 \footnote{http://www.statmt.org/wmt11/translation-task.html\#download}.   
\subsubsection{Preparing}
To prepare the tweets data, the system takes in total of 106 hashtag files, which includes both trial\_dir and train\_dir from the task, and put all tweets in one plain text file to form the tweet training corpus. Each tweet is on its own line. Be aware that during development phase of the system, we trained LMs solely on the train\_dir data, which includes 100 hashtag files, and tested it on the trial\_dir data consisting of 6 hashtag files. \\
For the news data, the system reads in all the sentences from the news files and again, put them in one giant plain text file to form the news training corpus. Each sentence takes its own line.
\subsubsection{Pre-processing}
In general, the pre-processing consists of two steps: filtering and tokenization. The filtering step is mainly for the tweet training corpus. Also, we applied various filtering and tokenziation combinations during development stage to determine the best settings (see section 4). 
\begin{itemize}
\item Filtering: the filtering process includes removing following elements from tweets:
\begin{itemize}
\item URLs
\item Twitter user names with symbol @ indicating the user name
\item Hashtags with symbol \# indicating the topic of the tweet
\end{itemize}
\item Tokenization: For both training data sets we splitted text by space and punctuation marks
\end{itemize}

\subsection{Language Model Training}
Once we have the corpora ready, we use the KenLM Toolkit to train the N-gram LMs on each corpus. LMs are estimated from the corpus using modified Kneser-Ney smoothing without pruning. KenLM reads in a plain text file and generates LMs in arpa format. We trained two different language models -- bigrams and trigrams -- for both training data sets. KenLM also implements back off technique, which simply applies the lower order N-gram's probability along with its back-off weights if the N-gram is not found. Instead of using the real probability of the N-gram, KenLM applies base 10 logarithm scheme. Here is an example arpa file of the trigram LM we trained on the tweets data:

\begin{table}[h!]
\begin{tabular}{ |l |c|}
\hline
N-gram 1=21580 \\
N-gram 2=60624 \\
N-gram 3=73837 \\
\hline
unigram:\\
-4.8225346   \textless unk\textgreater  0 \\
0   \textless s\textgreater  -0.47505832 \\
-1.4503417   \textless /s\textgreater  0 \\
-4.415446   Donner  -0.12937292 \\
...\\
\hline
bigrams:\\
...\\
-0.9799023  Drilling Gulf -0.024524588\\
...\\
\hline
trigrams:\\
...\\
-1.171928 I'll start thinking\\
...\\
\hline
\end{tabular}
\caption{Trigram LM on tweets data}
\label{table:1}
\end{table}
\noindent
Each N-gram line starts with the base 10 logarithm probability of that N-gram, followed by the N-gram which consists of N words. The base 10 logarithm of the back-off weight for the N-gram is followed after optionally. In this trigram LM trained on tweets data, there are 73,837 trigrams in total from the tweet training corpus. Notice that there are three "special" words in a language model: the beginning of a sentence denoted by \textless s\textgreater, the end of a sentence denoted by \textless /s\textgreater and the out of vocabulary word denoted by \textless unk\textgreater. In order to be able to handle the unknown words to estimate the probability of a tweet more accurately, in all our experiments we kept the \textless unk\textgreater word in our LMs. To figure out the best setting of language model for both tasks, we experimented using the language model with and without sentence boundaries.

\subsection{Tweet Scoring}
After training the N-gram model, the next step is scoring. For each hashtag file that needs to be evaluated, based on the trained N-gram LM, our system assigns a base 10 log probability as a score for each tweet in the hashtag file. The larger the score, the more likely of the tweet appears with the respect of that LM. Here is an example of scored tweet from hashtag file Bad\_Job\_In\_5\_Words.tsv based on the trigram LM trained on the tweets data:\\
\begin{table*}[h!]
\begin{tabular}{ |p{5cm}|p{5cm}|p{5cm}| } 
\hline
\multicolumn{3}{|c|}{The hashtag: \#BadJobIn5Words} \\
\hline
tweet\_id & tweet & score \\
\hline 
705511149970726912 & The host of Singled Out \#BadJobIn5Words @midnight & -19.923433303833008 \\
\hline
705538894415003648 & Donut receipt maker and sorter  \#BadJobIn5Words @midnight & -27.67446517944336 \\
\hline
\\
\caption{Scored tweet according to trigram LM. The format follows .tsv file provided by the task. The first column shows tweets\_id; the second column shows tweets; the third column shows the probability score computed based on the trigram LM. }
\label{table:2}
\end{tabular}
\end{table*}
\subsection{Tweet Prediction}
The system sorts tweets for each hashtag file based on their score in descending order, meaning the most probable one is listed on the top. For Task A, given a hashtag file, the system goes through the sorted list of tweets, compare each pair of tweets and produces a tsv format file as the task asks for. For each tweet pair twee\_1 and tweet\_2, if tweet\_1 has higher score, system outputs tweet\_ids for the pair followed by "1" and followed by "0" otherwise. For Task B, given a hashtag file, the systm simply outputs tweet\_ids in descending order of the sorted list. 


\section{Experiments and Results}
In this section we present the evaluation results of our system, as well as several post-evaluation experiments. Notice that the system used trigram LMs in evaluation. Both bigram and trigam LMs were used during system development and post-evaluation stage.

Table x.x shows results from developing stage. Note that for tweets data we trained language models on train\_dir data and tested on trial\_dir data. From this table We can tell the best setting to train language models for both data sets: for tweets data we decided to use trigrams and omit sentence boundaries (in this case, tweet boundaries); for news data we chose to train trigram language models on a tokenized news corpus.

\begin{table}[h!]
\begin{tabular}{ |p{0.5cm}|p{0.5cm}|p{0.5cm}|p{0.5cm}|p{0.5cm}|p{0.5cm}|p{0.7cm}|p{0.7cm}|}
\hline
DS & N & TS  & SB & LC & TK & AA & BD \\
\hline
t & 3 & T & F & F & F & 0.543 & 0.887 \\
\hline
t & 3 & T & T & T & F & 0.522 & 0.900 \\
\hline
t & 2 & T & F & F & F & 0.548 & 0.900 \\ 
\hline
n & 3 & NA & F & F & T & 0.539 & 0.923 \\
\hline
n & 3 & NA & F & F & F & 0.460 & 0.923 \\
\hline
n & 2 & NA & F & F & F & 0.470 & 0.900 \\
\hline
\end{tabular}
\caption{Development results. Abbreviations: for column names--"DS" stands for "Dataset", "N" stands for "N-gram", "SB" stands for "Sentence Boundaries", "LC" stands for "Lowercase", "TK" stands for "Tokenization", "AA" stands for "Subtask A Accuracy" and "BD" stands for "Subtask B Distance"; for table content--"t" stands for "tweets data", "n" stands for "news data", "T" stands for "true" which means "keep", "F" stands for "False" which means "discard". Notice that we kept three digits.s}
%\medskip
 
\label{table:3}
\end{table}
Since in development stage we implemented bigram and trigram LMs, we added bigram LMs in post-evaluation experiments. Table x.x shows the results of our system applying trigram LMs during evaluation along with bigram LMs results: 

\begin{table}[h!]
\begin{tabular}{ |p{0.5cm}|p{0.5cm}|p{0.5cm}|p{0.5cm}|p{0.5cm}|p{0.5cm}|p{0.7cm}|p{0.7cm}|}
\hline
DS & N & TS  & SB & LC & TK & AA & BD \\
\hline
t & 3 & T & F & F & F & 0.397 & 0.967 \\
\hline
t & 2 & T & F & F & F & 0.406 & 0.944 \\
\hline
n & 3 & NA & F & F & T & 0.627 & 0.872 \\
\hline
n & 2 & NA & F & F & T & 0.624 & 0.853 \\
\hline
\end{tabular}
\caption{Evaluation results and post-evaluation runs}
\label{table:4}
\end{table}
\section{Discussion and Future Work}
We believe that lack of tweets data could cause the failure on the language models comparing the amount of tweets data and news data we used. Therefore, one way to improve the system, especially the tweets data language model, is to collect more tweets that participate in the hashtag wars. We would also like to train news LMs using about as much text as we have for the tweets and see how the results compare. Additionally, we want to gather more news data and see if quantity of news training data would still make a difference. Besides applying N-gram language model approach to the task, we would also like to try some machine learning techniques, specifically deep learning method such as word2vec. From our perspective, deep learning method could play a role in this task in the sense of developing a system that learns humor from the show's point of view through neurons. It would also be interesting to see if even combining these two methods could help enhance the system.




% include your own bib file like this:
%\bibliographystyle{acl}
%\bibliography{acl2017}
\bibliography{semeval2017}
\bibliographystyle{acl_natbib}


\end{document}
